%%
%% This is file `sample-acmsmall.tex',
%% generated with the docstrip utility.
%%
%% The original source files were:
%%
%% samples.dtx  (with options: `acmsmall')
%% 
%% IMPORTANT NOTICE:
%% 
%% For the copyright see the source file.
%% 
%% Any modified versions of this file must be renamed
%% with new filenames distinct from sample-acmsmall.tex.
%% 
%% For distribution of the original source see the terms
%% for copying and modification in the file samples.dtx.
%% 
%% This generated file may be distributed as long as the
%% original source files, as listed above, are part of the
%% same distribution. (The sources need not necessarily be
%% in the same archive or directory.)
%%
%% Commands for TeXCount
%TC:macro \cite [option:text,text]
%TC:macro \citep [option:text,text]
%TC:macro \citet [option:text,text]
%TC:envir table 0 1
%TC:envir table* 0 1
%TC:envir tabular [ignore] word
%TC:envir displaymath 0 word
%TC:envir math 0 word
%TC:envir comment 0 0
%%
%%
%% The first command in your LaTeX source must be the \documentclass command.
\documentclass[acmsmall,review, nonacm]{acmart}
%% NOTE that a single column version is required for 
%% submission and peer review. This can be done by changing
%% the \doucmentclass[...]{acmart} in this template to 
%% \documentclass[manuscript,screen]{acmart}
%% 
%% To ensure 100% compatibility, please check the white list of
%% approved LaTeX packages to be used with the Master Article Template at
%% https://www.acm.org/publications/taps/whitelist-of-latex-packages 
%% before creating your document. The white list page provides 
%% information on how to submit additional LaTeX packages for 
%% review and adoption.
%% Fonts used in the template cannot be substituted; margin 
%% adjustments are not allowed.
%%
%% \BibTeX command to typeset BibTeX logo in the docs
\AtBeginDocument{%
  \providecommand\BibTeX{{%
    \normalfont B\kern-0.5em{\scshape i\kern-0.25em b}\kern-0.8em\TeX}}}

%% Rights management information.  This information is sent to you
%% when you complete the rights form.  These commands have SAMPLE
%% values in them; it is your responsibility as an author to replace
%% the commands and values with those provided to you when you
%% complete the rights form.
% \setcopyright{acmlicensed}
% \copyrightyear{2024}
% \acmYear{2024}
% \acmDOI{XXXXXXX.XXXXXXX}

%%
%% The majority of ACM publications use numbered citations and
%% references.  The command \citestyle{authoryear} switches to the
%% "author year" style.
%%
%% If you are preparing content for an event
%% sponsored by ACM SIGGRAPH, you must use the "author year" style of
%% citations and references.
%% Uncommenting
%% the next command will enable that style.
%%\citestyle{acmauthoryear}

%% Code listing commands
\usepackage{listings}
\usepackage{xcolor}

% Theme is everforest light from: https://gogh-co.github.io/Gogh/
\definecolor{codegreen}{RGB}{167, 192, 128}
\definecolor{codeblack}{RGB}{75, 86, 91}
\definecolor{codegray}{RGB}{92, 106, 114}
%\definecolor{codepurple}{rgb}{0.58,0,0.82}
\definecolor{backcolour}{RGB}{253, 246, 227}
\definecolor{codeblue}{RGB}{57, 148, 197}


\lstdefinestyle{mystyle}{
    backgroundcolor=\color{backcolour},   
    commentstyle=\color{codegreen},
    keywordstyle=\color{codeblue},
    numberstyle=\tiny\color{codegray},
    %stringstyle=\color{codepurple},
    basicstyle=\ttfamily\scriptsize,
    breakatwhitespace=false,         
    breaklines=true,                 
    captionpos=b,                    
    keepspaces=true,                 
    numbers=left,                    
    numbersep=5pt,                  
    showspaces=false,                
    showstringspaces=false,
    showtabs=false,                  
    tabsize=2,
    xleftmargin=0.3cm,
    frame=tlbr,  
    framerule=0pt,
}

\lstset{style=mystyle}

%%
%% end of the preamble, start of the body of the document source.
\begin{document}

%%
%% The "title" command has an optional parameter,
%% allowing the author to define a "short title" to be used in page headers.
\title{Binary Analysis for Missed Vectorization Opportunities Detection}

%%
%% The "author" command and its associated commands are used to define
%% the authors and their affiliations.
%% Of note is the shared affiliation of the first two authors, and the
%% "authornote" and "authornotemark" commands
%% used to denote shared contribution to the research.
\author{Amos Herz}
\email{amherz@ethz.ch}
\author{Alessandro Legnani}
\email{alegnani@ethz.ch}

\affiliation{%
  \institution{ETH Zürich}
  \city{Zürich}
  \country{Switzerland}
  \postcode{8092}
}

%%
%% By default, the full list of authors will be used in the page
%% headers. Often, this list is too long, and will overlap
%% other information printed in the page headers. This command allows
%% the author to define a more concise list
%% of authors' names for this purpose.
\renewcommand{\shortauthors}{Herz and Legnani}

%%
%% The abstract is a short summary of the work to be presented in the
%% article.
\begin{abstract}
  
\end{abstract}

%%
%% The code below is generated by the tool at http://dl.acm.org/ccs.cfm.
%% Please copy and paste the code instead of the example below.
%%
% TODO: ?
% \begin{CCSXML}
% <ccs2012>
%  <concept>
%   <concept_id>00000000.0000000.0000000</concept_id>
%   <concept_desc>Do Not Use This Code, Generate the Correct Terms for Your Paper</concept_desc>
%   <concept_significance>500</concept_significance>
%  </concept>
%  <concept>
%   <concept_id>00000000.00000000.00000000</concept_id>
%   <concept_desc>Do Not Use This Code, Generate the Correct Terms for Your Paper</concept_desc>
%   <concept_significance>300</concept_significance>
%  </concept>
%  <concept>
%   <concept_id>00000000.00000000.00000000</concept_id>
%   <concept_desc>Do Not Use This Code, Generate the Correct Terms for Your Paper</concept_desc>
%   <concept_significance>100</concept_significance>
%  </concept>
%  <concept>
%   <concept_id>00000000.00000000.00000000</concept_id>
%   <concept_desc>Do Not Use This Code, Generate the Correct Terms for Your Paper</concept_desc>
%   <concept_significance>100</concept_significance>
%  </concept>
% </ccs2012>
% \end{CCSXML}

% \ccsdesc[500]{Do Not Use This Code~Generate the Correct Terms for Your Paper}
% \ccsdesc[300]{Do Not Use This Code~Generate the Correct Terms for Your Paper}
% \ccsdesc{Do Not Use This Code~Generate the Correct Terms for Your Paper}
% \ccsdesc[100]{Do Not Use This Code~Generate the Correct Terms for Your Paper}

%%
%% Keywords. The author(s) should pick words that accurately describe
%% the work being presented. Separate the keywords with commas.
% \keywords{Do, Not, Us, This, Code, Put, the, Correct, Terms, for,
%   Your, Paper}

% \received{20 February 2007}
% \received[revised]{12 March 2009}
% \received[accepted]{5 June 2009}

%%
%% This command processes the author and affiliation and title
%% information and builds the first part of the formatted document.
\maketitle

\section{Introduction}
Modern compilers are often able to automatically vectorize code using 
SIMD instructions.
Take, as an example, the following code snippet: 

\lstinputlisting[caption={\texttt{copy.c}}, label={lst:copy}, language=C]{listings/copy.c}

\noindent We can compile it with the following set of compiler flags
\begin{itemize}
  \item \texttt{-O3}: tells the compiler to use the highest level
        of optimization available.
  \item \texttt{-fno-tree-loop-distribute-patterns}:
        prevents replacing the loop with a call to \texttt{memcpy}
  \item \texttt{-fno-tree-vectorize}: prevents vectorization
\end{itemize}

\noindent Which will produce the following assembly code:

\lstinputlisting[caption={\texttt{copy.c} compiled with vectorizations disabled}, label={lst:copy_compiled}]{listings/copy.s}

\noindent However, by compiling without the \texttt{-fno-tree-vectorize} flag,
the compiler will produce the following vectorized code (note the use of wider instructions and registers):

\lstinputlisting[caption={\texttt{copy.c} compiled with vectorizations enabled}, label={lst:copy_vectorized}]{listings/copy_vectorized.s}

\noindent Nonetheless, autovectorization is \textit{difficult}, compilers tend to miss many optimizations (as shown by \citet{Feng2021})
, and more than often
vectorizing a small piece of code requires large changes to the whole code base 
(as was done for example by \citet{Chen22}).
The main goal of this project is to develop a method for identifying missed opportunities 
for vectorization in existing code. That is, given an existing binary, we want to identify 
loops that could be vectorized but are not.

\section{Approach}
The idea is to use a dynamic analysis framework (such as Intel Pin\footnote{https://software.intel.com/sites/landingpage/pintool/docs/98830/Pin/doc/html/index.html}, 
DynamoRIO\footnote{https://dynamorio.org/} or angr\footnote{https://angr.io/})to performs a dynamic analysis 
that traces an execution and identifies vectorizable code. 
Given the trace of execution of a program (instructions and memory accesses), we can perform a dataflow analysis to identify 
``parallel'' computations. 



%%
%% The acknowledgments section is defined using the "acks" environment
%% (and NOT an unnumbered section). This ensures the proper
%% identification of the section in the article metadata, and the
%% consistent spelling of the heading.
% \begin{acks}

% \end{acks}

%%
%% The next two lines define the bibliography style to be used, and
%% the bibliography file.
\bibliographystyle{ACM-Reference-Format}
\bibliography{report-references}

%%
%% If your work has an appendix, this is the place to put it.
% \appendix

% \section{Research Methods}

% \subsection{Part One}


% \subsection{Part Two}


% \section{Online Resources}


\end{document}
\endinput
%%
%% End of file `sample-acmsmall.tex'.
